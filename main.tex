%%----------LaTeX template for teachers-----------------------------%%
%%----------Samuel S. Watson----------------------------------------%% 
%%----------January 2013--------------------------------------------%% 

% http://math.mit.edu/~sswatson/latexforteachers.html

\documentclass[12pt]{article} % Specifies font size
%----------------PACKAGES-------------------------------------------%%
\usepackage[margin=1in]{geometry} % Sets all four margins to 1 inch
\usepackage[pdftex]{graphicx} % Allows inclusion of image files
\usepackage{amssymb} % Access to extra math symbols
\usepackage{amsmath} % Access to extra math symbols
\usepackage{wrapfig} % Allows wrapping of text around figures
\usepackage{calc} % Gives access to a basic calculator 
\usepackage{array} % Gives custom commands for array environments 
\usepackage{anyfontsize} % Lets you adjust the font size easily
%-------------------------------------------------------------------%%

%----------------COMMANDS-------------------------------------------%%
\newcommand\blank{\underline{\hspace{2cm}}} % Gives a blank 
\newcounter{prob} % A new counter for current problem number
\setcounter{prob}{1} % Start the counter at the value 1
\newcommand\itm{
\fbox{\textbf{\theprob}} \refstepcounter{prob}
} % Calls problem number
\newcommand{\problem}[1]{\makebox[0.5cm]{\itm}   
  \begin{minipage}[t]{\textwidth-0.5cm} #1 \end{minipage} 
} % An environment for a problem statement on or more lines
\newcommand{\pairofprobs}[2]{
  \begin{minipage}[t]{0.5\textwidth}\itm #1 \end{minipage} 
  \begin{minipage}[t]{0.5\textwidth}\itm #2 \end{minipage} 
} % Fits two problems on a line
\newcommand{\threeprobs}[3]{
\begin{minipage}[t]{0.31\textwidth}\itm #1 \end{minipage} \hfill
 \begin{minipage}[t]{0.31\textwidth}\itm #2 \end{minipage} \hfill 
 \begin{minipage}[t]{0.31\textwidth}\itm #3 \end{minipage}
} % Fits three problems on a line
\newcommand{\fourprobs}[4]{
\begin{minipage}[t]{0.21\textwidth}\itm #1 \end{minipage} \hfill
 \begin{minipage}[t]{0.21\textwidth}\itm #2 \end{minipage} \hfill 
 \begin{minipage}[t]{0.21\textwidth}\itm #3 \end{minipage} \hfill 
 \begin{minipage}[t]{0.21\textwidth}\itm #4 \end{minipage}
} % Fits four problems on a line


\newcounter{choice} % Counter for multiple choice problems 
\setcounter{choice}{1} % Start the counter at the value 1
\newcommand\achoice{
(\alph{choice}) \stepcounter{choice}
} % Generates letter for multiple choice option
\newcommand{\answers}[5]{\vspace*{-7mm} 
  \begin{tabular}{l@{\hspace{1mm}}p{0.9\textwidth}}
    \achoice & #1 \\ \achoice & #2 \\ \achoice & #3 \\ 
    \achoice & #4 \\ \achoice & #5 \end{tabular}
  \setcounter{choice}{1}
} % Makes multiple-choice options 
%---------------------------------

% The commands below are for setting up arithmetic 
% problems with the four basic operations. See examples 
% in the CONTENT section  

\newcommand\divi[2]{
#1 \: \begin{array}{|l}
\hline #2
\end{array}
}

\newcommand\mult[2]{
$\begin{array}{rr} 
 & #1 \\ 
 \times & #2 \\ \hline 
 \end{array}$}
 
\newcommand\addi[2]{
  $\begin{array}{rr} 
   &  #1 \\ 
    + & #2 \\ \hline 
  \end{array}$}

\newcommand\subt[2]{
  $\begin{array}{rr}
    & #1 \\ 
    - & #2 \\ \hline
  \end{array}$}
%-------------------------------------------------------------------%%

%-----------FORMATTING----------------------------------------------%%
\pagestyle{empty} % Ensures that no page numbers are printed
\parskip = 0.2 in % Puts a little space between paragraphs 
\parindent = 0.0 in % Enforces no indentation for paragraphs
%-------------------------------------------------------------------%%

%-----------USAGE EXAMPLES------------------------------------------%%
% To try any of the examples below, uncomment them and paste 
% them below the \begin{document} command in the CONTENT section. 

% To set up a division problem such as 93 divided by 3:
% \divi{3}{93}

% To set up a muliplication problem such as 14 times 4:
% \mult{14}{4}

% To put two problems on the same line: 
% \pairofprobs{\divi{3}{93}}{\mult{14}{4}} 

% To include a 3cm vertical space between questions 
% \vspace{3cm} 

%--------------------------------------------------------------------%%

%-----------CONTENT--------------------------------------------------%%
\begin{document}

{\fontsize{14}{17}\selectfont
%%%%%%%%%%%%%%%%%% Part 1
\begin{center}
  \textsc{ YPLS Homework for February 26}
\end{center}

{$\textbf{Set A}$.} 
Use the order of operations ($\textbf{BEDMAS}$) to evaluate the following.

\threeprobs
{$2+12\div 4 -1 $}
{$1+3^2-3\times2 $}
{$1+2\times 4 - 2^2$}

\threeprobs
{$(3-2)(2-3)\times 4$}
{$(1+2+3+4)^2 \div 5 $}
{$(-1)^2 - (9-8)^2 $}

\setcounter{prob}{1}

{$\textbf{Set B}$.}  Simplify each expression by collecting like-terms.

\threeprobs
{$2x -3y + 5x$}
{$2(a-b) + b$}
{$x +xy + yx -y$}

\threeprobs
{$x^2 +x + -3x + x^2$}
{$a + ba + 3(ab +a)$}
{$3(2x-4y)-2(x-y)$}

\setcounter{prob}{1}

{$\textbf{Set C}$.} Solve each equation to find the unknown value.

\threeprobs
{$x - 4 = 10$}
{$3y=12$}
{$\cfrac{2x+ 10}{x} = 4$}

\threeprobs
{$\cfrac{x}{3} -7 = -2$}
{$6b+2 =44$}
{$\cfrac{a}{10} = \cfrac{3}{2}+a$}

\setcounter{prob}{1}

{$\textbf{Set D}$.} Expand and simplify each expression.

\threeprobs
{$x(2x+1) -x$}
{$(x+3)(x-2)$}
{$(2x - 3)^2$}

\threeprobs
{$2y(3y-1) +2y$}
{$(y-5)(y+5)$}
{$(x^2+x+1)(2x-1)$}

\setcounter{prob}{1}

{$\textbf{Set E}$.} Evaluate if $x=-2$.

\threeprobs
{$3x + 2$}
{$x^2-x+1$}
{$-x + 1$}

%\setcounter{prob}{1}
%{$\textbf{Set F}$.} Factor each expression %fully.

%\threeprobs
%{$x^2 + 2x + 1$}
%{$9x^2-4$}
%{$x^2 +5x +6$}

} % end font size setting

\newpage
\setcounter{prob}{1}
{\fontsize{14}{17}\selectfont
%%%%%%%%%%%%%%%%%% Part 3
\begin{center}
  \textsc{ Word Problems}
\end{center}

Use the strategy of your choice to solve these word problems.

\problem{Dante has twice as many jelly beans as Carlos. Natalia has 4 more jelly beans than Carlos and Dante together. All together, Dante, Carlos, and Natalia have 58 jelly beans. How many jelly beans does Carlos have?}

\problem{A tank is $\frac{2}{5}$ filled with water. When another $26$ liters of water are poured in, the tank becomes $\frac{5}{6}$ full. Find the capacity of the tank.}

\problem{A piece of rope 3m 66cm long was cut into 2
pieces. The longer piece was twice as long as the shorter piece. What was the length of the longer piece?}

\problem{Samy had 130 stickers and Devi had 50 stickers.
After Samy gave Devi some stickers, Samy had twice as many stickers as Devi. How many stickers did Samy give to Devi?} 

\problem{Ali saved twice as much as Ramat. Maria saved
\$60 more than Ramat. If they saved \$600 altogether, how much did
Maria save?}

\problem{$\frac{1}{3}$ of the beads in a box are red, $\frac{2}{3}$ of the remainder are blue and the rest are yellow. If there are $24$ red beads, how many yellow beads are there?}

\problem{Minghua bought a bag of marbles. $\frac{1}{4}$ of the
marbles were blue, $\frac{1}{8}$ were green and $\frac{1}{5}$ of the remainder were yellow. If there were $24$ yellow marbles, how many marbles did he buy?} 

\problem{Rahim has 30\% more books than Gopal. If Rahim has 65 books, how many books does Gopal have?}

\problem{John and Mary had \$350 altogether. After John
spent $\frac{1}{2}$ of his money and Mary spent $\frac{1}{3}$ of her money, they each had an equal amount of money left. How much did they spend altogether?}

}



\end{document}
%--------------------------------------------------------------------%%